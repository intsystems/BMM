\documentclass{beamer}
\usetheme{Boadilla}

\usepackage{amsmath}
\usepackage{amsfonts}



\title{Wishart Distribution}
\author{Konstantin Yakovlev}
\institute{MIPT, 2022}


\begin{document}

\begin{frame}
    \titlepage
\end{frame}


\begin{frame}
    \tableofcontents
\end{frame}


\section{Definition}

\begin{frame}{Definition}
    \begin{block}{Definition}
    Let $\{\mathbf{x}_i\}_{i=1}^n, \mathbf{x}_i \in \mathbb{R}^p$ be $\mathrm{i.i.d.}$ samples from
    $\mathcal{N}(\mathbf{x}|\boldsymbol{0}, \mathbf{\Sigma})$. Then
    \[
        \mathbf{M} = \sum_{i=1}^n \mathbf{x}_i\mathbf{x}_i^\top \sim \mathcal{W}_p(\mathbf{\Sigma}, n),
    \]
    where $n$ is a number of degrees of freedom, $\mathbf{\Sigma}$ is a scale matrix.
    \end{block}

    Note that when $p = 1$, it is easy to see that $\mathbf{M} \sim \chi^2_{n}$, i.e.
    $\mathcal{W}_1(1, n) =^d \chi^2_n$.

\end{frame}


\section{Properties}

\begin{frame}{Properties}
    \begin{block}{Property 1}
        Let $\mathbf{M} \sim \mathcal{W}_p(\mathbf{M}|\mathbf{\Sigma}, n)$.
        Let $\mathbf{A}\in \mathbb{R}^{q\times p}$ be a a constant matrix.
        Then
        \[
            \mathbf{A}\mathbf{M}\mathbf{A}^\top \sim \mathbf{W}_q(\mathbf{A}\mathbf{\Sigma}\mathbf{A}^\top, n).
        \]
    \end{block}
    \begin{proof}
        Rewrite the definition of $\mathbf{A}\mathbf{M}\mathbf{A}^\top$:
        \[
            \mathbf{A}\mathbf{M}\mathbf{A}^\top = \sum_{i=1}^n\mathbf{A}\mathbf{x}_i\mathbf{x}_i^\top\mathbf{A}^\top.
        \]
        We know that $\mathbf{y} = \mathbf{A}\mathbf{x} \sim \mathcal{N}(\mathbf{y}|\mathbf{0}, \mathbf{A}\mathbf{\Sigma}\mathbf{A}^\top)$. Then
        \[
            \sum_{i=1}^n\mathbf{y}_i\mathbf{y}_i^\top \sim \mathcal{W}_q(.|\mathbf{A}\mathbf{\Sigma}\mathbf{A}^\top, n).
        \]
    \end{proof}

\end{frame}


\begin{frame}{Properties}
    \begin{block}{Cochran’s theorem}
        Let $\mathbf{P} \in \mathbb{R}^{n\times n}$ be a projection matrix
        ($\mathbf{P} = \mathbf{P}^2$) of rank $r$.
        Let $\mathbf{X} = [\mathbf{x}_1^\top, \ldots, \mathbf{x}_n^\top]^\top \in \mathbb{R}^{n\times p}$.
        Rows are $\mathrm{i.i.d.}$.
        Each row $\mathbf{x}_i \sim \mathcal{N}(\mathbf{x}_i|\mathbf{0}, \mathbf{\Sigma})$.
        Let $\mathrm{rank}\mathbf{\Sigma} = p$. Then
        \[
            \mathbf{X}^\top\mathbf{P}\mathbf{X} \sim \mathcal{W}(.|\mathbf{\Sigma}, r).
        \]
        \[
            \mathbf{X}^\top(\mathbf{I} - \mathbf{P})\mathbf{X} \sim \mathcal{W}(.|\mathbf{\Sigma}, n-r).
        \]
        Moreover, $\mathbf{X}^\top\mathbf{P}\mathbf{X}$ and $\mathbf{X}^\top(\mathbf{I} - \mathbf{P})\mathbf{X}$ are
        indepandent.
    \end{block}

    \begin{block}{Property 2}
    Let $\{\mathbf{x}_i\}_{i=1}^n$ be $\mathrm{i.i.d.}$ samples from
    $\mathcal{N}(\mathbf{x}|\boldsymbol\mu, \mathbf{\Sigma})$. Then
    \[
        n\mathbf{S} = \sum_{i=1}^n(\mathbf{x}_i - \overline{\mathbf{x}})(\mathbf{x}_i - \overline{\mathbf{x}})^\top
        \sim \mathcal{W}_p(.|\mathbf{\Sigma}, n - 1).
    \]
    \end{block}
\end{frame}


\begin{frame}{Proof of property 2}
    Let $\mathbf{P} = \mathbf{I} - \frac{1}{n}\mathbf{1}\mathbf{1}^\top \in \mathbb{R}^{n\times n}, \mathrm{rank}\mathbf{P} = 1$.
    Consider $\mathbf{P}^2 = \mathbf{I} + \frac{1}{n^2}\mathbf{1}\mathbf{1}^\top\mathbf{1}\mathbf{1}^\top
    - \frac{2}{n}\mathbf{1}\mathbf{1}^\top = \mathbf{P}$.
    
    Now consider $\mathbf{X}^\top\mathbf{P}\mathbf{X} = \sum_{i=1}^n\mathbf{x}_i\mathbf{x}_i^\top -\frac{1}{n}\mathbf{X}^\top\mathbf{1}\mathbf{1}^\top\mathbf{X} = n\mathbf{S}$.
    Here we used the fact that $\overline{\mathbf{x}} = \frac{1}{n}\mathbf{X}^\top\mathbf{1}$.

    Hence, using Cochran's theorem,
    \[
        n\mathbf{S} \sim \mathcal{W}(.|\mathbf{\Sigma}, n - 1).
    \]
\end{frame}


\begin{frame}{Hotelling's $T^2$ dustribution}
    \begin{block}{Definition}
        Let $\mathbf{x} \sim \mathcal{N}(x|\mathbf{0}, \mathbf{I}), \mathbf{x}\in \mathbb{R}^p$.
        Let $\mathbf{M} \sim \mathcal{W}_p(\mathbf{M}|\mathbf{I}, n), \mathbf{M} \in \mathbb{R}^{p\times p}$.
        $\mathbf{M}$ and $\mathbf{x}$ are independent. Then
        \[
            n\mathbf{x}^\top\mathbf{M}^{-1}\mathbf{x} \sim T^2(p, n).
        \]
    \end{block}

    \begin{block}{Property 3}
        Let $\mathbf{x} \sim \mathcal{N}(\boldsymbol{x}|\boldsymbol{\mu}, \mathbf{\Sigma})$,
        $\mathbf{M} \sim \mathcal{W}_p(\mathbf{M}|\mathbf{\Sigma}, n)$ are independent.
        Let also $\mathrm{det}\mathbf{\Sigma} \not=0$. Then
        \[
            n(\mathbf{x} - \boldsymbol\mu)^\top\mathbf{M}^{-1}(\mathbf{x} - \boldsymbol\mu) \sim T^2(p, n).
        \]
    \end{block}
    \begin{proof}
        Let $\mathbf{y} = \mathbf{\Sigma}^{-1/2}(\mathbf{x} - \mathbf{\mu})$ is a standard normal random vector.
        Let $\mathbf{Z} = \mathbf{\Sigma}^{-1/2}\mathbf{M}\mathbf{\Sigma}^{-1/2} \sim \mathcal{W}_p(\mathbf{Z}|\mathbf{I}, n)$.
        Then $n\mathbf{y}^\top\mathbf{Z}\mathbf{y} \sim T^2(p, n)$. 
    \end{proof}
\end{frame}


\begin{frame}{Properties}
    \begin{Theorem}
        \[
            \frac{n - p}{(n - 1)p}n\mathbf{x}^\top\mathbf{M}^{-1}\mathbf{x} \sim F_{p, n-p},
        \]
        where $F_{., .}$ is a Fisher distribution.
    \end{Theorem}

    \begin{Corollary}
        From property 2, property 3 and Cochran's theorem follows that
        \[
            \frac{n - p}{p}(\overline{\mathbf{x}} - \boldsymbol\mu)\mathbf{S}^{-1}(\overline{\mathbf{x}} - \mathbf{\mu})
            \sim F_{p, n - p}.
        \]
    \end{Corollary}
\end{frame}


\section{Statistical Tests}

\begin{frame}{Statistical tests}

    \begin{block}{Known $\mathbf{\Sigma}$}
        Given $\{\mathbf{x}_i\}_{i=1}^n$ from $\mathcal{N}(\mathbf{x}|\boldsymbol\mu, \mathbf{\Sigma}),
        \mathbf{x} \in \mathbb{R}^p$.

        $H_0: \quad \boldsymbol\mu = \boldsymbol\mu_0$

        $H_1: \quad \boldsymbol\mu <\not= > \boldsymbol\mu_0$

        $T(\mathbf{X}) = n(\overline{\mathbf{x}} - \boldsymbol\mu_0)^\top\mathbf{\Sigma}^{-1}(\overline{\mathbf{x}} - \boldsymbol\mu_0)$

        Null distribution: $\chi^2_p$.
    \end{block}

    \begin{block}{Unknown $\mathbf{\Sigma}$}
        Given $\{\mathbf{x}_i\}_{i=1}^n$ from $\mathcal{N}(\mathbf{x}|\boldsymbol\mu, \mathbf{\Sigma}),
        \mathbf{x} \in \mathbb{R}^p$.

        $H_0: \quad \boldsymbol\mu = \boldsymbol\mu_0$

        $H_1: \quad \boldsymbol\mu <\not= > \boldsymbol\mu_0$

        $T(\mathbf{X}) = \frac{n-p}{p}(\overline{\mathbf{x}} - \boldsymbol\mu_0)^\top\mathbf{S}^{-1}(\overline{\mathbf{x}} - \boldsymbol\mu_0)$.

        Null distribution: $F_{p, n-p}$

    \end{block}

\end{frame}


\section{Fast Sampling}

\begin{frame}{Fast Sampling}
    \begin{block}{Odell and Feiveson, 1966}
        Let $\{V_i\}_{i=1}^p$ be independent random variables: $V_i \sim \chi^2_{p - i + 1}$. Let $\{N_{ij}\}_{ij=1}^n$
        be indepentent standard normal random variables idendependent of $V_i$. Define $b_{ij} = b{ji}$ for $i, j \in \overline{1,p}$:
        \[
            b_{ii} = V_i + \sum_{r=1}^{i-1}N^2_{ri},
        \]
        \[
            b_{ij} = N_{ij}\sqrt{V_i} + \sum_{r=1}^{i-1}N_{ri}N_{rj}.
        \]
    Then $\mathbf{B} = \{b_{ij}\} \sim \mathcal{W}_p(\mathbf{I}, n)$.
    \end{block}
    Note that here we have to sample $O(p^2)$ stardard normal random variables instead of $O(np)$
    is naive sampling.
\end{frame}


\begin{frame}{Literature}
    \begin{enumerate}
        \item \textbf{Definition, properties and tests} https://rich-d-wilkinson.github.io/MATH3030/7-multinormal.html
        \item \textbf{Fast sampling} https://www.math.wustl.edu/~sawyer/hmhandouts/Wishart.pdf
    \end{enumerate}
\end{frame}



\end{document}
