\documentclass{beamer}
\usetheme{Boadilla}

\usepackage{amsmath}
\usepackage{amsfonts}
\usepackage{hyperref}
\usepackage{biblatex}
\addbibresource{biblio.bib}

\newcommand\norm[1]{\left\lVert#1\right\rVert}
\newcommand\normx[1]{\left\Vert#1\right\Vert}


\title{Your Classifier is Secretly an Energy Based Model and You Should Treat It Like One}
\author{Maria Kovaleva}
\institute{MIPT, 2022}


\begin{document}

\begin{frame}
    \titlepage
\end{frame}


\begin{frame}
    \tableofcontents
\end{frame}


\section{Basic Concept}

\begin{frame}{Basic Concept of EBM}
    \textbf{Energy-Based Models} (EBMs) capture dependencies by associating a scalar energy to each configuration of the variables
    
    \textbf{Learning:} finding an energy function that associates low energies to correct values, and higher energies to incorrect values
    
    \textbf{Inference:} finding зфкфьуеукы that minimize the energy
    
    \textbf{Pros:} no requirement for proper normalization, in comparison with probabilistic models
    
     \begin{block}{Probability density}
    $p(\mathbf{x})$ for $\mathbf{x}\in \mathbb{R}^{D}$:
    $ p_{\theta}(\mathbf{x}) = \frac{\exp{(-E_{\theta}(\mathbf{x}))}}{Z(\theta)}$, where $E_{\theta}(\mathbf{x})$ is energy function and $Z(\theta) = \int_{\mathbf{x}} \exp{(-E_{\theta}(\mathbf{x}))} d \mathbf{x}$ is normalizing constant or partition function
    \end{block}
    
   
\end{frame}

\begin{frame}{Learning of EBM}
    \begin{block}{}
        Estimate $Z(\theta)$ can be challenging for most energy functions.
        Derivative of the log-likelihood for a single example x with respect to $\theta$: $\frac{\partial \log p_{\theta}(\mathbf{x})}{\partial \theta} = \mathop{\mathbb{E}}_{p_{\theta}(\mathbf{x'})} \left[\frac{\partial E_{\theta}(x')}{\partial \theta}\right] - \frac{\partial E_{\theta}(x)}{\partial \theta}$
    \end{block}
    
    \begin{block}{Learning} 
         Approximate the expectation in derivative of the log-likelihood using a sampler based on Stochastic Gradient Langevin Dynamics (SGLD):
        
        $\mathbf{x}_0 \sim p_0(\mathbf{x})$, $\mathbf{x}_{i+1} = \mathbf{x}_{i} - \frac{\alpha}{2} \frac{\partial E_{\theta}(\mathbf{x}_i)}{\partial \mathbf{x}_i} + \epsilon$, $\epsilon \sim \mathcal{N}(0,\alpha)$, 
        
        where $p_0(\mathbf{x})$ is typically a Uniform distribution,  $\alpha$ - step-size (should be decayed following a polynomial schedule). 
        
        Practically $\alpha$ and $\epsilon$ is often chosen separately leading to a biased sampler which allows for faster training
    \end{block}
\end{frame}

\section{What Your Classifier Is Hiding}
\begin{frame}{In classifiers}
    \textbf{Classification:} D parameters, K classes, parametric function, $f_{\theta} : \mathbb{R}^{D} \rightarrow \mathbb{R}^{K}$ return logits
   
    \textbf{Categorical distribution} parameterized by   $f_{\theta}$ via Softmax transfer function: $p_{\theta}(y|\mathbf{x}) = \frac{\exp{(f_{\theta}(\mathbf{x})[y])}}{\sum_{y'}{\exp{(f_{\theta}(\mathbf{x})[y'])}}}$

\end{frame}


\begin{frame}{ Reinterpretation of the logits}
    Let's define an energy based model with $E_{\theta}(\mathbf{x},y) = - f_{\theta}(\mathbf{x})[y]$ and unknown normalizing constant $Z_{\theta}$ :
    $$p_{\theta}(\mathbf{x}, y) = \frac{\exp{(f_{\theta}(\mathbf{x})[y])}}{Z(\theta)}$$
    Then 
    $$p_{\theta}(\mathbf{x}) = \sum_{y}p_{\theta}(\mathbf{x}, y) = \frac{\sum_{y}{\exp{(f_{\theta}(\mathbf{x})[y])}}}{Z(\theta)}$$
    \begin{block}{Conclusion}
        The LogSumExp(·) of the logits of any classifier can be re-used to define the energy function as $E_{\theta}(\mathbf{x}) = −LogSumExp(f_{\theta}(\mathbf{x})[y]) = -\log \sum_{y}{\exp{f_{\theta}(\mathbf{x})[y] }}$ 
        
        A generative model hidden within every standard discriminative model! 
    \end{block}
\end{frame}

\begin{frame}{Optimization}
    \begin{block}
        $\log p_{\theta}(\mathbf{x}, y) = \log p_{\theta}(\mathbf{x}) + \log p_{\theta}(y|\mathbf{x})$
       
        Where:
        
        \textbf{1.}  $p(y|\mathbf{x})$ optimized using standard cross-entropy
        
        \textbf{2.}  $\log p(\mathbf{x})$ optimized with SGLD where gradients are taken with respect to $LogSumExp(f_{\theta}(\mathbf{x})[y])$ as it described for EBMs
    \end{block}
    \begin{block}{Remarks}
            \textbf{1.}  We don't know $Z_{\theta}$, because of it we use SLGD sampler
            
            \textbf{2.}  The estimator of $\log p(\mathbf{x})$ will be biased when using a MCMC sampler with a finite number of steps
    \end{block}
\end{frame}

\section{Results}
\begin{frame}{Experiment}
    CIFAR10, CIFAR100, SVHN datasets were used
    
    All architectures used are based on Wide Residual Networks where we have removed batch-normalization to ensure that our models’ outputs are deterministic functions of the input
    
    Comparison with Resudal Flow\footnote{Ricky TQ Chen, Jens Behrmann, David Duvenaud, and Jorn-Henrik Jacobsen. Residual flows for invertible generative modeling}, Glow\footnote{Durk P Kingma and Prafulla Dhariwal. Glow: Generative flow with invertible 1x1 convolutions}, IGEBM\footnote{Yilun Du and Igor Mordatch. Implicit generation and generalization in energy-based models}, SNGAN\footnote{Takeru Miyato, Toshiki Kataoka, Masanori Koyama, and Yuichi Yoshida. Spectral normalization for generative adversarial networks}, NCSN\footnote{Yang Song and Stefano Ermon. Generative modeling by estimating gradients of the data distribution} and other models
    
    
\end{frame}

\begin{frame}{Results}
    \begin{enumerate}
        \item Performance rivaling the state of the art in both discriminative and generative modeling achieved
        \item Increase in inception scores (IS) and Frechet Inception Distance (FID) for generative modeling
        \item Increase in accuracy for discriminative problem
        \item Decrease in Expected Calibration Error (ECE) for calibration \footnote{A classifier is considered calibrated if its predictive confidence, $\max_y p(y|\mathbf{x})$, aligns with its misclassification rate. Thus, when a calibrated classifier predicts label y with confidence $0.9$ it should have a $90\%$ chance of being correct.}
        \item Better out-of-distribution (OOD) detection \footnote{Out-of-distribution (OOD) detection is a binary classification problem. $s_{\theta} \in \mathbb{R}$ We desire that the scores for in-distribution examples are higher than that out-of-distribution examples} results for $s_{\theta}(x) = \max_y p_{\theta}(y|\mathbf{x})$ and for $s_{\theta} = - \norm{\frac{\partial \log p_{\theta}(\mathbf{x})}{\partial \mathbf{x}}}_2$
        \item Better robustness
    \end{enumerate}
    
\end{frame}

\begin{frame}{Literature}
\nocite{*}
\printbibliography
\end{frame}

\end{document}