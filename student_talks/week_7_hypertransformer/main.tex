\documentclass{beamer}
\usetheme{Boadilla}

\usepackage{amsmath}
\usepackage{amsfonts}
\usepackage{hyperref}

\usepackage{amsmath}
\DeclareMathOperator*{\argmax}{arg\,max}
\DeclareMathOperator*{\argmin}{arg\,min}

\title{HyperTransformer}
\author{Timofey Chernikov}
\institute{MIPT, 2023}


\begin{document}

\begin{frame}
    \titlepage
\end{frame}


% \begin{frame}
%     \tableofcontents
% \end{frame}


\section{Motivation \& Background}
\begin{frame}{Motivation}
    \begin{block}{Problem statement}
        Given a broad set of tasks, develop a single model that will be able to quickly specialize for a particular task with little data.
    \end{block} 
\end{frame}


\begin{frame}{Existing approaches}
 \begin{block}{Metric-Based Learning}
    One family of approaches involves mapping input samples into an embedding space and
    then using some nearest neighbor algorithm to label query
    samples based on the distances from their embeddings to
    embeddings of labeled support samples. The metric used to
    compute the distances can either be the same for all tasks, or
    can be task-dependent
    \end{block}
\end{frame}

\section{Theory}
\begin{frame}{Formal definition}
    \begin{block}{}
        A set of tasks $\{t| \in \mathcal{T}\}$
        
        Loss $\mathcal{L}(f,t)$ that quantifies the correctness of any
        model $f$ attempting to solve $t$.

        A task description $\tau(t)$ that is sufficient for communicating this task and finding
        the optimal model that solves it,
        includes any available information like labeled and
        unlabeled samples, image metadata, textual descriptions, etc.

        The weight generation algorithm can then be viewed as a
        method of using a set of training tasks $\mathcal{T}_{train}$ for discovering
        a particular solver aφ that given $\tau(t)$ for a task $t$ similar to
        those present in the training set, produces an optimal model
        $f* = a_\phi(\tau) \in \mathcal{F}$ minimizing $\mathcal{L}(f*, t)$

        $$ \underset{\arg \min}{\phi in \Phi} \mathbb{E}_{t \sim p(t)} \mathcal{L}(a_\phi(\tau(t)), t) $$

    \end{block}
\end{frame}


\section{Model}
\begin{frame}{Model scheme}
    \begin{block}{}
        \includegraphics[scale=0.25]{model_scheme.png}
    \end{block}
\end{frame}


\section{Model}
\begin{frame}{Model description}
    \begin{block}{}
        A solver $a_\phi$ is the core of a few-shot learning algorithm -
        it encodes the knowledge of the training task distribution within its weights $\phi$. 
        It's a Transformer-based model that takes a task description $\tau$ and produces weights for some or all
        layers $\{\theta_l | l \in [1, L]\}$ of the generated CNN model. Layer
        weights that are not generated are instead learned end-to-end together 
        with HT weights as ordinary task-agnostic variables. 
        In other words, these learned layers are modified
        during the training phase and remain static during the evaluation phase. 
        In our experiments generated CNN models contain a set of convolutional 
        layers and a final fully-connected logits layer. The
        weights are generated layer-by-layer starting from the first
        layer: $\theta_1(\tau ) \rightarrow \theta_2 (\theta_1; \tau ) \rightarrow ... \rightarrow \theta_L(\theta_1,...,\theta_{L-1}; \tau)$.
    \end{block}
\end{frame}

\section{Model}
\begin{frame}{Results}
    \includegraphics[scale=0.19]{results.png}
\end{frame}




\begin{frame}{Literature}
    \begin{enumerate}
        \item \textbf{Main article} \href{https://proceedings.mlr.press/v162/zhmoginov22a/zhmoginov22a.pdf}
        {HyperTransformer: Model Generation for Supervised and Semi-Supervised
        Few-Shot Learning}.
    \end{enumerate}
\end{frame}



\end{document}